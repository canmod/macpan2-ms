\documentclass[12pt]{article}\usepackage[]{graphicx}\usepackage[]{xcolor}
% maxwidth is the original width if it is less than linewidth
% otherwise use linewidth (to make sure the graphics do not exceed the margin)
\makeatletter
\def\maxwidth{ %
  \ifdim\Gin@nat@width>\linewidth
    \linewidth
  \else
    \Gin@nat@width
  \fi
}
\makeatother

\definecolor{fgcolor}{rgb}{0.345, 0.345, 0.345}
\newcommand{\hlnum}[1]{\textcolor[rgb]{0.686,0.059,0.569}{#1}}%
\newcommand{\hlstr}[1]{\textcolor[rgb]{0.192,0.494,0.8}{#1}}%
\newcommand{\hlcom}[1]{\textcolor[rgb]{0.678,0.584,0.686}{\textit{#1}}}%
\newcommand{\hlopt}[1]{\textcolor[rgb]{0,0,0}{#1}}%
\newcommand{\hlstd}[1]{\textcolor[rgb]{0.345,0.345,0.345}{#1}}%
\newcommand{\hlkwa}[1]{\textcolor[rgb]{0.161,0.373,0.58}{\textbf{#1}}}%
\newcommand{\hlkwb}[1]{\textcolor[rgb]{0.69,0.353,0.396}{#1}}%
\newcommand{\hlkwc}[1]{\textcolor[rgb]{0.333,0.667,0.333}{#1}}%
\newcommand{\hlkwd}[1]{\textcolor[rgb]{0.737,0.353,0.396}{\textbf{#1}}}%
\let\hlipl\hlkwb

\usepackage{framed}
\makeatletter
\newenvironment{kframe}{%
 \def\at@end@of@kframe{}%
 \ifinner\ifhmode%
  \def\at@end@of@kframe{\end{minipage}}%
  \begin{minipage}{\columnwidth}%
 \fi\fi%
 \def\FrameCommand##1{\hskip\@totalleftmargin \hskip-\fboxsep
 \colorbox{shadecolor}{##1}\hskip-\fboxsep
     % There is no \\@totalrightmargin, so:
     \hskip-\linewidth \hskip-\@totalleftmargin \hskip\columnwidth}%
 \MakeFramed {\advance\hsize-\width
   \@totalleftmargin\z@ \linewidth\hsize
   \@setminipage}}%
 {\par\unskip\endMakeFramed%
 \at@end@of@kframe}
\makeatother

\definecolor{shadecolor}{rgb}{.97, .97, .97}
\definecolor{messagecolor}{rgb}{0, 0, 0}
\definecolor{warningcolor}{rgb}{1, 0, 1}
\definecolor{errorcolor}{rgb}{1, 0, 0}
\newenvironment{knitrout}{}{} % an empty environment to be redefined in TeX

\usepackage{alltt}

\input{preamble}

\title{McMasterPandemic: Lessons for Software Development from the COVID-19 Pandemic (TODO: working title)}
\author{Jonathan Dushoff$^1$ \and Benjamin M. Bolker$^2$ \and David J.\,D.\ Earn$^2$ \and Michael Li \and Irena Papst \and (TODO: others) \and Steve C Walker$^2$ (TODO: order is currently a placeholder) \\
  $^1$Such-and-such University;\\
  $^2$Department of Mathematics and Statistics,\\
  McMaster University, Hamilton, Ontario, Canada, L8S 4K1
  \\
  \medskip
  \spewmoddates
}



\IfFileExists{upquote.sty}{\usepackage{upquote}}{}
\begin{document}

\maketitle

\linenumbers

\begin{abstract}
  TBD
\end{abstract}

%%\begin{keywords}
%%epidemics, SIR model, \dots
%%\end{keywords}

%%\begin{MSCcodes}
%%34E05, 34E13, 37N25, 92D30
%%\end{MSCcodes}

\section{Introduction}\label{sec:intro}

The \texttt{McMasterPandemic} compartmental model and associated \texttt{R} package were developed to provide forecasts and insights to public health agencies throughout the COVID-19 pandemic. Much was learned about developing general purpose compartmental modelling software during this experience, but the pressure to deliver regular forecasts made it difficult to focus on the software itself. Here we describe the \texttt{macpan2} package -- a re-imagining of \texttt{McMasterPandemic}, building it from the ground up with architectural and technological decisions that address the many lessons that we learned from COVID-19 about software.

Impactful applied public health modelling requires many interdisciplinary steps along the path from epidemiological research teams to operational decision makers. Researchers must quickly tailor a model to an emerging public-health concern, validate and calibrate it to data, work with decision makers to define model outputs useful for stakeholders, configure models to generate those outputs, and package up those insights in an appropriate format for stakeholders. Unlike traditional modelling approaches, \texttt{macpan2} tackles this challenge from a software-engineering perspective, which allows us to systematically address bottlenecks along this path to impact in ways that will make future solutions easier to achieve. The goal is to enable researchers to focus on their core strengths and fill knowledge gaps efficiently and effectively.

Although \texttt{macpan2} is designed as a compartmental modelling tool that is agnostic about the underlying computational engine, it currently uses template model builder as the sole engine. Template model builder (TMB) is an R modelling package based on a C++ framework incorporating mature automatic differentiation and matrix algebra libraries.

\section{Methods}\label{sec:methods}

This is how to solve everything:

Many \LaTeX\ packages and macros are loaded when you
\verb|\input{preamble.tex}|, which is done in \texttt{ms.tex}.

Rather than \macro{ref} or \macro{eqref}, use \macro{cref} (from the
\texttt{cleveref} package) for referring to equations, figures,
sections, appendices.  For example, use \verb|\cref{sec:results}| to
get \cref{sec:results} and \verb|\cref{eq:everything,fig:everything,app:ToDo}|
to get \cref{eq:everything,fig:everything,app:ToDo}.

Use \macro{linenumbers} to get line numbers, and comment out this
command if you don't want line numbers.  To ensure you get line numbers
in paragraphs that include an \texttt{align} environment, enclose in a
\verb|linenomath*| environment like so:
\begin{verbatim}
\begin{linenomath*}
\begin{align}\label{eq:everything}
  U = 0,
\end{align}
\end{linenomath*}
\end{verbatim}
This example generated \cref{eq:everything} below.

\subsection{Fancy symbols}

\david{This subsection is in progress\dots}

Developing good notation is often a serious challenge.
For example, \cite{ParsEarn24} used $\Xyb{0}{i}(y)$ for
an approximation to $X(y)$ within the $y$-axis
boundary layer.  The \texttt{scalerel} package is very helpful
for constructing complicated symbols in a way that scales
appropriately
when they are used in subscripts.   The macros used
in \cite{ParsEarn24} produced $A_{\Xyb{0}{i}}$, whereas
using the \macro{scaleobj} macro from \texttt{scalerel} yields
$A_{\Xybfancy{0}{i}}$ $A_{\Xybfancy{\Xybfancy{0}{i}}{i}}$
$A_{\Xybfancy{\scaleobj{0.8}{\Xybfancy{0}{i}}}{i}}$

\newcommand{\tmpout}{{x_{A}}}
\begin{equation}
\tmpout_{\tmpout_{\tmpout_{\tmpout}}}
\end{equation}
\renewcommand{\tmpout}{{x_{\scaleobj{0.8}{A}}}}
\begin{equation}
\tmpout_{\tmpout_{\tmpout_{\tmpout}}}
\end{equation}

\section{Results}\label{sec:results}

The complete equation for everything is
\begin{linenomath*}
\begin{align}\label{eq:everything}
  U = 0,
\end{align}
\end{linenomath*}
where $U$ is for ``universe''.  This equation is illustrated
in \cref{fig:everything}.  Remarkably, it does not depend on $\R_0$,
at least not explicitly.  Even more remarkably, \cref{eq:everything}
was not mentioned in some mathematical epidemiology review articles
\cite{Earn+02,Earn04,Earn08,Earn09}.

\begin{knitrout}
\definecolor{shadecolor}{rgb}{0.969, 0.969, 0.969}\color{fgcolor}\begin{kframe}
\begin{alltt}
\hlstd{a} \hlkwb{<-} \hlkwd{sqrt}\hlstd{(}\hlnum{2}\hlopt{*}\hlstd{pi)}
\end{alltt}
\end{kframe}
\end{knitrout}

\begin{figure}
  \begin{center}
    \Huge EVERYTHING
    %%\includegraphics[width=0.95\textwidth]{fig_everything.pdf}
\begin{knitrout}
\definecolor{shadecolor}{rgb}{0.969, 0.969, 0.969}\color{fgcolor}
\includegraphics[width=\maxwidth]{figure/example.figure-1} 
\end{knitrout}
  \end{center}
  \caption{This figure illustrates \cref{eq:everything}.}
  \label{fig:everything}
\end{figure}

Note that $\sqrt{2\pi}=2.5066283$ as calculated in the
\texttt{example.calculation} chunk above.  You can supress
the chunk in the typeset document with the chunk option
\texttt{echo=FALSE}.

\section{Discussion}\label{sec:discussion}

Everything is explained by \cref{eq:everything,fig:everything}.

\section*{Acknowledgements}

DJDE was supported by an NSERC Discovery Grant.

\bibliography{ms}
\bibliographystyle{vancouver}

%%\input{mstables}

%%\FloatBarrier

%%\input{msfigures}

\appendix

\section{To Do}\label{app:ToDo}

\begin{itemize}
\item appendices
\item tables
\item Rnw
\item journal templates
\end{itemize}

\end{document}
